% Copyright (c) 2015 Mihail Atanassov
% Edited by Stewart Smith 2015
%
% Permission is hereby granted, free of charge, to any person obtaining a copy
% of this software and associated documentation files (the "Software"), to deal
% in the Software without restriction, including without limitation the rights
% to use, copy, modify, merge, publish, distribute, sublicense, and/or sell
% copies of the Software, and to permit persons to whom the Software is
% furnished to do so, subject to the following conditions:
%
% The above copyright notice and this permission notice shall be included in
% all copies or substantial portions of the Software.
%
% THE SOFTWARE IS PROVIDED "AS IS", WITHOUT WARRANTY OF ANY KIND, EXPRESS OR
% IMPLIED, INCLUDING BUT NOT LIMITED TO THE WARRANTIES OF MERCHANTABILITY,
% FITNESS FOR A PARTICULAR PURPOSE AND NONINFRINGEMENT. IN NO EVENT SHALL THE
% AUTHORS OR COPYRIGHT HOLDERS BE LIABLE FOR ANY CLAIM, DAMAGES OR OTHER
% LIABILITY, WHETHER IN AN ACTION OF CONTRACT, TORT OR OTHERWISE, ARISING FROM,
% OUT OF OR IN CONNECTION WITH THE SOFTWARE OR THE USE OR OTHER DEALINGS IN
% THE SOFTWARE.

%%== MISSION STATEMENT ==%%

% Uncomment the following lines if you're adding your own mission statement from a PDF, change the number of pages in the options for \includepdf.

%\addcontentsline{toc}{chapter}{Mission Statement}

%\includepdf[pagecommand={\thispagestyle{headings}}, pages=1-2]{mission_statement.pdf}


%if you're adding your own file comment out or delete all the lines below.
\chapter*{Mission Statement}\label{mission_statement}
\addcontentsline{toc}{chapter}{Mission Statement}

\section*{Background}
\addcontentsline{toc}{section}{Background}
Simulation Program with Integrated Circuit Emphasis or SPICE has now been widely used in the IC design and verification. Solving of sparse matrices often takes up most of the SPICE simulation time. Lower–upper (LU) decomposition is the most commonly used method to solve matrices. It factorizes a matrix into two factors – a lower triangular matrix \emph{L} and an upper triangular matrix \emph{U}. In this way, we only need to solve triangular systems to get results. However, the sparse-matrix computation is hard to parallelize on regular processors due to the irregular structure of the matrices. Modern FPGAs, however, have the potential to compute these hard-to-parallelise problems more efficiently due to its flexible reconfigurability.

\section*{Aim \& Tasks}
\addcontentsline{toc}{section}{Aim \& Tasks}
Implement LU Decomposition on FPGAs and accelerate it.
\begin{itemize}
  \item To implement LU decomposition with C++.
  \item	To implement LU decomposition on FPGA.
  \item	Test the power consumption and efficiency of the algorithm on FPGA.
  \item	Accelerate the LU decomposition by parallelization and make it robust enough for circuit simulation application.
  \item	Produce a parallel FPGA-based Matrix Solver.
\end{itemize}

\section*{Background Knowledge}
\addcontentsline{toc}{section}{Background Knowledge}
\begin{itemize}
  \item	C++ is designed with an orientation towards system programming and embedded. It offers a direct hardware mapping and can be synthesized into FPGA. Modern C++ has also provided parallel version of many algorithms.

  \item Verilog is a hardware description language. It is commonly used in design and verification of digital circuit in register-transfer level and can be physically realized by synthesis software.

\end{itemize}

\section*{Resources}
\addcontentsline{toc}{section}{Resources}
\begin{itemize}
  \item Xilinx Vitis\\
  The Vitis unified software platform is a free new tool that combines all aspects of Xilinx software development into one unified environment. It enables accelerated applications on various Xilinx platforms including FPGAs. It is easy to get from Xilinx official website.
  \item Intel oneAPI\\
  oneAPI is a cross-industry, open, standards-based unified programming model that delivers a common developer experience across accelerator architectures. It provides optimized compilers, libraries, frameworks and analysis tools for development on CPUs, GPUs and FPGAs. It provides free access through Intel official website.

\end{itemize}


\pagebreak
\vspace{36pt}
\newcommand*\justify{%
  \fontdimen2\font=0.4em% interword space
  \fontdimen3\font=0.2em% interword stretch
  \fontdimen4\font=0.1em% interword shrink
  \fontdimen7\font=0.1em% extra space
\hyphenchar\font=`\-% allowing hyphenation
}

% Add the following to create a stand-alone mission statement for the start of
% the project.
%   \begin{sloppypar}
%     \texttt{\justify
%       The supervisor and student are satisfied that this project is
%       suitable for performance and assessment in accordance with the
%       guidelines of the course documentation.}
%
%     \vspace{2em}
%     \texttt{Signed\hspace{\fill}}
%
%     \vspace{1em}
%     \texttt{Student:\hspace{\fill}\makebox[0.5\linewidth]{\dotfill}}
%
%     \vspace{1em}
%     \texttt{Industrial
%       Supervisor:\hspace{\fill}\makebox[0.5\linewidth]{\dotfill}}
%
%     \vspace{1em}
%     \texttt{University
%       Supervisor:\hspace{\fill}\makebox[0.5\linewidth]{\dotfill}}
%
%     \vspace{1em}
%     \texttt{Date:\hspace{\fill}\makebox[0.35\linewidth]{\dotfill}}
%   \end{sloppypar}
